% !TeX spellcheck = en_EN
%cioa
\documentclass[a4paper,12pt]{article}
\usepackage[a4paper]{geometry}
\usepackage[utf8]{inputenc}
\usepackage[english]{babel}
\usepackage{subcaption}
\usepackage{amsmath}
\usepackage{amsfonts}
\usepackage{siunitx}
\usepackage{float}
\usepackage{pdfpages}
\usepackage{booktabs} 
\usepackage{pgfpages}

%for printing two page per face
\pgfpagesuselayout{2 on 1}[a4paper,landscape] 
\usepackage[bottom]{footmisc}

\newenvironment{sistema}%
{\left\lbrace\begin{array}{@{}l@{}}}%
	{\end{array}\right.}

\title{Mechatronic system simulation}

\begin{document}
	{ \huge
		Mechatronic System Simulation}
	\footnote{Version 1.3 \ \ \ \today}
	\section{LAPLACE TRANSFORM}
	Definition:
	\begin{equation}
		\lim_{\varepsilon \rightarrow 0^+}\lim_{M \rightarrow \infty }\int_{-\varepsilon}^{M} f(t)e^{st} \ dt = \int_{0^-}^{\infty } f(t)e^{st} \ dt
	\end{equation}

	The solution of an ODE in the Laplace domain is in the form: $SOL=\frac{P(s)}{Q(s)}$ with usually deg(P(s)) $<$ deg(Q(s)). If this condition is not respected and deg(P(s)) $>$ deg(Q(s)) we could use the polynomial division to transform the solution in the form $SOL=\frac{P(s)}{Q(s)}=\frac{T(s) Q(s) + R(s)}{Q(s)} = T(s) + \frac{R(s)}{Q(s)}$ with deg(R(s)) $<$ deg(Q(s)).\\
	According to the root of Q(s) are necessary differential strategies to perform the PFE.\\ \\
	\fbox{ \parbox{\textwidth}{
			Example of poly division:
			\begin{gather*}
				\frac{P(s)}{Q(s)}=\frac{s^4+1}{s^2+1} = (s^2-1) \ (s^2+1)+2
			\end{gather*}
			\begin{table}[H]
				\centering
				\begin{tabular}{ccccc|c}
					$s^4$ & 0 & 0 & 0 & 1 & $s^2+0s+1$\\
					\toprule
					$s^4$ &  & $s^2$ &  &  & $s^2-1$\\ \midrule
					&  & $-s^2$ &  & 1 & \\ \midrule
					&  & $-s^2$ &  & -1 & \\ \midrule
					&  &  &  & 2 & \\ 
				\end{tabular}
	\end{table}}}
	\subsection{Real single roots}
	\begin{gather}
		G(s)=\frac{b_0 +b_1 s+...+b_m s^m}{(s-p_1) (s-p_2)...(s-p_n)} = \frac{\alpha_1}{s-p_1} +\frac{\alpha_2}{s-p_2}+...+\frac{\alpha_n}{s-p_n}
		\label{Real_single_roots}\\
		\alpha_i = \lim_{ s \to p_i} (s-p_i) G(s)
		\label{Real_single_roots_coeff}
	\end{gather}
	\fbox{ \parbox{\textwidth}{
			Example:
			\begin{gather*}
				\frac{s}{(s-1)(s+3)(s-4)}=\frac{A}{s-1}+\frac{B}{s+3}+\frac{C}{s-4}\\
				1) \ \text{Multiply both side for (s-1) and evaluate for s=1: } \\
				A=\frac{s}{(s+3)(s-4)}=\frac{1}{(1+3)(1-4)}=-\frac{1}{12}\\
				2) \ \text{Multiply both side for (s+3) and evaluate in s=-3: }\\
				B=\frac{s}{(s-1)(s-4)}=\frac{-3}{28}\\
				3) \ \text{Multiply both side for (s-4) and evaluate in s=4: } \\
				C=\frac{s}{(s-1)(s+3)}=\frac{4}{21}
	\end{gather*}}}
	
	
	\subsection{Real multiple roots}
	\begin{gather}
		G(s)=\frac{b_0 +b_1 s+...+b_m s^m}{(s-p)^n} = \frac{\alpha_1}{s-p} +\frac{\alpha_2}{(s-p)^2}+...+\frac{\alpha_n}{(s-p)^n} 
		\label{Real_multiple_roots}\\
		\alpha_{n-k}=\frac{1}{K!} \lim_{ s \to p} \frac{d^k}{d^k s}((s-p)^n G(s))
		\label{Real_multiple_roots_coeff}
	\end{gather}
	
	\fbox{ \parbox{\textwidth}{
			Example
			\begin{gather*}
				\frac{1+s}{(s-4)^4}=\frac{A}{s-4}+\frac{B}{(s-4)^2}+\frac{C}{(s-4)^3}+\frac{D}{(s-4)^4}\\
				1) \ \text{Multiply both side for $(s-4)^4$ and evaluate for s=4: }\\
				1+s = D + (s-4)(SOMETHING) \Rightarrow D=5 \\
				2) \ \text{Derive both side of 1 ) and evaluate in s=4}\\
				1 = 3 A (s-4)^2 +2B (s-4) +C \Rightarrow C=1\\
				3) \ \text{Derive both side of 2 )and evaluate in s=4}\\
				0 = 6 A (s-4) +B  \Rightarrow B=0\\
				4) \ \text{Derive both side of 3 )and evaluate in s=4}\\
				0 = 6 A \Rightarrow A=0
	\end{gather*}}}
	\subsection{Single + multiple roots}
	\fbox{ \parbox{\textwidth}{
			Example
			\begin{gather*}
				\frac{1+s}{(s+1)(s+2)^3}=\frac{A}{s-1}+\frac{B}{s+2}+\frac{C}{(s+2)^2}+\frac{D}{(s+2)^3}\\
				1) \ \text{Multiply both side for $(s+1)(s+2)^3$ and evaluate for s=1}\\
				1+s = A (s+2)^3 + (s-1) (SOMETHING); 1+1=A(3)^3 \Rightarrow A=2/27\\
				2) \ \text{Applying the \textit{deflaction} (Substitute A's value and isolate on left side)} \\
				\frac{-2s^3-12s^2+3s+11}{27}=(s-1) \left[ B (s+2)^2 +C(s+2) +D \right] \\
				3) \ \text{Perform the polynomial division} \\
				\frac{-2s^2-14s-11}{27}=B(s+2)^2+C(s+2)+D \ \ (1)\\
				4) \ \text{Evaluate (1) in s=-2} \Rightarrow D=1/3\\
				5) \ \text{Derive (1) and evaluate at s=-2}\\
				\frac{-4s-14}{27}=2 B (s+2) +C \ \ (2) \Rightarrow C=-2/9\\
				6) \ \text{Derivate (2) and collect}\\
				\frac{-4}{27}=2B \Rightarrow B=-2/27\\
				7) \text{We get}\\
				\mathcal{L}^{-1} \left\{ \frac{2/27}{s-1}+\frac{-2/27}{s+2}+\frac{-2/9}{(s+2)^2}+\frac{1/3}{(s+2)^3} \right\}=\frac{2}{27} e^t-e^{-2t} \left( \frac{2}{27}-\frac{2}{9} t
				+\frac{1}{3 \cdot 2}\frac{1}{s^2} \right)
	\end{gather*}}}
	
	\subsection{Complex single roots}
	\textbf{Method 1} \\
	Observation: We have $\frac{P(s)}{Q(s)}$. If $\alpha \in \mathbb{C}$ is a root also $\overline{\alpha}$ is it (Q($\alpha$)=Q($\overline{\alpha}$)=0).\\
	\begin{gather}
		\frac{P(s)}{Q(s)}= \frac{A}{s-\alpha}+\frac{\overline{A}}{s-\overline{\alpha}}=\frac{s(A+\bar A)-A \bar \alpha - \bar A \alpha}{s^2-s(\alpha + \bar \alpha) + \mid \alpha \mid ^2}
		\label{Complex_single_roots}\\
		A=\frac{-i P(\alpha)}{2 Im \{ \alpha \} }
		\label{Complex_single_roots_coeff}
	\end{gather}
	In order to inverse the Laplace transfor we have to trasform (\ref{Complex_single_roots}) to get a polynomial with real coefficients at the denominator (and then to apply the laplace transfor of sin and cos functions).
	
	\begin{gather}
		\frac{s(A+\bar A)-A \bar \alpha - \bar A \alpha}{s^2-s(\alpha + \bar \alpha) + \mid \alpha \mid ^2} = 
		\frac{s C + ( D \omega -C a)}{(s-a)^2+\omega^2}
		\label{Complex_single_roots_ver2}\\
		\text{Match the denominator: } a=Re\{ \alpha \} \ \ \ \omega= Im\{ \alpha \} \\
		\text{Match the numerator: } C=A + \bar A=2 Re\{ A \} \ \ \ D=\frac{C a - A \bar \alpha - \bar A \alpha}{\omega}
		\label{Complex_single_roots_coeff_ver2}
	\end{gather}
	Once the coefficients are found they could be used in the inverse Laplace trasform, remembering that:
	\begin{equation}
		\mathcal{L}^{-1} \left\{ C \frac{s-a}{(s-a)^2+ \omega ^2} + D \frac{ \omega }{(s-a)^2+ \omega ^2} \right\} (t) = C e^{a t} \cos(\omega t) + D e^{a t} \sin(\omega t) 
		\label{Complex_single_roots_coeff_ver3}
	\end{equation}
	\textbf{Method 2} \\
	Another method is treated the multiple complex roots as single roots. The disadvantage of these method is to have to work with complex numbers.\\
	For the final transformation the Euler's formulas could be usefull: 
	\begin{gather}
		e^{it}=\cos(t)-i\sin(t)
		\label{Euler_1}\\
		sin(t)=\frac{e^{it}-e^{-it}}{2i} \ \ \ cos(t)=\frac{e^{it}+e^{-it}}{2}
		\label{Euler_2}
	\end{gather}
	\subsection{Multiple complex roots}
	\begin{gather}
		\frac{P(s)}{(s-\alpha)^2 (s- \bar \alpha)^2} = \frac{A}{(s-\alpha)} +\frac{B}{(s-\alpha)^2}+\frac{\bar A}{(s- \bar \alpha)} +\frac{\bar B}{(s- \bar \alpha)^2}
		\label{Multiple_complex_roots}\\
		B=\frac{P(\alpha )}{4 Im \left\{\alpha \right \}^2}
		\label{Multiple_complex_roots_coeff}
	\end{gather}
	Multiply both sides of (\ref{Multiple_complex_roots}) for the denominator of the left hand side and collect the known terms:
	\begin{gather}
		P(s) - B(s -\bar \alpha)^2 -B(s-\alpha)^2=A(s-\alpha)(s-\bar \alpha)^2+ \bar A (s-\bar \alpha) (s- \alpha)^2
		\label{Multiple_complex_roots_coeff_ver2}
	\end{gather}
	Derive both sides and evaluate it for s=$\alpha$. After the calculation the result is in the form:
	\begin{gather}
		A=\frac{NUMBER}{4 Im\left\{\alpha\right\}^2}
	\end{gather}
	The part of PFE $\frac{A}{(s-\alpha)}+\frac{\bar A}{(s- \bar \alpha)}$ could be managed as the case of simple root. For the other part we can use the rule of the inverse Laplace trasform:
	\begin{equation}
		\mathcal{L}^{-1} \left\{ \frac{C}{(s-z)^{k+1}} + \frac{\bar C}{(s- \bar z)^{k+1}} \right\} (t) = t^k \mathcal{L}^{-1} \left\{ \frac{C}{s-z} + \frac{\bar C}{s- \bar z} \right\} (t) 
		\label{Multiple_complex_roots_coeff_ver3}
	\end{equation}
	\textbf{General case}\\
	Starting from:
	\begin{gather}
		\frac{P(s)}{(s^2-2as+b)^r Q(s)}
	\end{gather}
	There are two main possible PFE.\\
	In the case with r=2: 
	\begin{gather}
		- \frac{P(s)}{(s^2-2as+t)^2 Q(s)}= \frac{A'(t)s+B'(t)}{s^2-2as+t}-\frac{A(t)s+B(t)}{(s^2-2as+t)^2}\frac{q(s,t)}{Q(s)} \notag{}\\
		\text{with:  } q(s,t)=\frac{d q(s,t)}{dt}
		\label{multiple_conj_1}
	\end{gather}
	In case with r=3:
	\begin{gather}
		\frac{P(s)}{(s^2-2as+t)^3 Q(s)}=\frac{A(t)s+B(t)}{(s^2-2as+t)^3}-\frac{A'(t)s+B'(t)}{(s^2-2as+t)^2}+\frac{A''(t)s+B''(t)}{2(s^2-2as+t)}+\frac{q_H(s,t)}{2 Q(s)} \notag\\
		\text{with:  } q_H=\frac{d^2q}{dt^2}
	\end{gather}
	To find $q(s,t)$ and $q_H(s,t)$ we could multiply both sides for the denominator of the LHS and evaluate at $s^2=2as-t$. \\
	Then we have to pass through an auxiliary function to get the coefficients A(t) and B(t):
	\begin{equation}
		\frac{P(s)}{(s^2-2as+t) Q(s)}=\frac{A(t)s+B(t)}{s^2-2as+t}+\frac{q(s,t)}{Q(s)}
	\end{equation}
	\fbox{ \parbox{\textwidth}{
			Example
			\begin{gather*}
				\frac{s^2+2s+3}{(s^2+2s+2)(s^2+2s+5)}\\
				1) \ \text{Rewrite the  the transformation of sine and cosine}\\
				\frac{s^2+2s+3}{(s^2+2s+2)(s^2+2s+5)}=\frac{As+B}{s^2+2s+2}+\frac{Cs+D}{s^2+2s+5}\\
				2) \ \text{Multiply both side for the denominator of the LHS}\\
				s^2+2s+3=(As+B)(s^2+2s+5)+(Cs+D)(s^2+2s+2)\\
				3) \ \text{Performing all the calculations, the coefficients could be matched}\\
				A=0 \ \ B=\frac{1}{3} \ \ C=0 \ \ \ D=\frac{2}{3}
				4) \ \text{Now we get}\\
				\frac{\frac{1}{3}}{s^2+2s+2}+\frac{\frac{2}{3}}{s^2+2s+5} \\
				5) \ \text{Transform every term of $4)$ as the sum of sine and cosine}\\
				\frac{\frac{1}{3}}{s^2+2s+2} = \frac{C_1(s-a_1) +  D_1 \omega_1 }{(s-a_1)^2+\omega_1^2}\\
				\frac{\frac{2}{3}}{s^2+2s+5} = \frac{C_2 (s-a_2) +  D_2 \omega_2 }{(s-a_2)^2+\omega_2^2}\\
				6) \ \text{Perform the anti-transformation with rules 15 and 16 }			
	\end{gather*}}}
	\newpage
	\section{Function Minimization with KKT conditions}
	The problem is in the form:
	\begin{gather*}
		\text{Minimize} \ \ \ f( \stackrel{-}{x})
		\label{fun_to_min}\\
		\text{Subject to} \ \ \ h_{i}(\stackrel{-}{x})=0 \ \ \ and \ \ \ \ g_{j}(\stackrel{-}{x})>0 \ \ \ (i=1,...,m; j=0,...,p)
	\end{gather*}
	Steps for the solution:
	\begin{enumerate}
		\item Write the Lagrangian function
		\begin{equation}
			\mathcal{L}(\stackrel{-}{x},\stackrel{-}{\lambda},\stackrel{-}{\mu}) = f( \stackrel{-}{x})- \sum_{k=1}^{m}\lambda_k h_k(\stackrel{-}{x})- \sum_{k=1}^{p}\mu_k g_k(\stackrel{-}{x})
		\end{equation}
		\item Write the non-linear system. The solutions fulfil the \textbf{first order necessary conditions} to be a minimum point: 
		\begin{equation}
			\begin{sistema}
				\nabla_{\stackrel{-}{x}}\mathcal{L}(\stackrel{-}{x})=0\\
				\mu g(\stackrel{-}{x})=0\\
				\mu \ge 0\\
				g_K(\stackrel{-}{x}) \ge 0\\
				h_K(\stackrel{-}{x})=0
			\end{sistema}
		\label{first_necessary}
		\end{equation}
		
		
		\item Solve the non linear system and get the candidate points:
		\begin{equation}
			P_{i}(\stackrel{-}{x},\stackrel{-}{\lambda},\stackrel{-}{\mu}) \ \ \ \text{with} \ \ \ i=1,...,r
		\end{equation}
		\item Verify if the constraints are actives, i.e. those for which $g_{i}(P_{j})$=0. If there are not active constrain then the kernel is the identity and we can directly study the lagrangian. Active constraints should be \textit{qualified} so they must be linearly independent (ie the matrix should have maximum rank).
		\item For the combinations of point-function which satisfy the condition of the previous point calculate the gradient of the constraint 
		\begin{equation}
			\nabla_{\stackrel{-}{x}} g_K(P_i)
		\end{equation}
		\item Compute the values of $ \ \nabla_{\stackrel{-}{x}} h_K(P_i)$
		\item Put in a matrix the results of the two previous points
		\begin{align}
			H = \begin{bmatrix}
				\nabla_{\stackrel{-}{x}} g_K(P)\\
				\nabla_{\stackrel{-}{x}} h_K(P_i)
			\end{bmatrix}
		\end{align}
		\item Calculate
		\begin{equation}
			H \ w = 0
		\end{equation}
		the system could be not univocally determined and there can be more than one free parameters. The kernel k is a matrix with a number of row equal to the dimension of $\stackrel{-}{x}$ and a number of column equal to the number of free parameters of the solution. In column there are vector who holds to the solution of the system.
		\item Verify the nature of the point
		\begin{equation}
			A=k^T\nabla_{\stackrel{-}{x}}^2\mathcal{L}(P)k
		\end{equation}
		Cases:
		\begin{itemize}
			\item det(A) $\ge$ 0 is a \textbf{necessary} condition to be a minimum
			\item det(A) $=$ 0 $\rightarrow$ nothing could be said about P. To solve the problem we can use the $\varepsilon$ trick: if
			\begin{equation}
				\lim_{\varepsilon \to 0} \frac{d}{d\varepsilon}det(A+\varepsilon I) > 0
			\end{equation}
			then P is a minimum
			\item det(A) $<$ 0 $\rightarrow$ P is not a minimum
			\item det(A) $>$ 0 $\rightarrow$ P is a minimum (sufficient condition to be a minimum)\\
			This condition could be also verified by checking that all the determinants of the principal minors are positive (ie for the Sylvester's theorem).
		\end{itemize}
	\end{enumerate}
	\textbf{REMARK}:
	\begin{itemize}
		\item First order necessary conditions: satisfy (\ref{first_necessary}), and have qualified constraints
		\item Second order necessary condition:   det(A) $\ge$ 0 (Attention to the question's formulation!)
		\item Second order sufficient condition:   det(A) $>$ 0 (Attention to the question's formulation!)
	\end{itemize}
	\fbox{ \parbox{\textwidth}{
			Example
			\begin{gather*}
				\text{minimize} f(x,y,z)=z-yz^2\\
				\text{subject to: } x=y , \ \ 0 \le y \le 1	\\
			\end{gather*}
			\begin{enumerate}
				\item Rewrite the constraints as equality and inequality with 0
				\begin{gather*}
				x=y \Rightarrow h(x,y,z) x-y=0\\
				0 \le y \Rightarrow g_1(x,y,z) = y \ge 0\\
				y \le 1 \Rightarrow g_2(x,y,z) = 1-y \ge 0
				\end{gather*}
				\item Write the lagrangian
				\begin{gather*}
				\mathcal{L} = f(x,y,z)- \lambda (x-y)- \mu_1 (y) -\mu_2 (1-y)
				\end{gather*}
				\item Write the nonlinear system		
			\[
			\begin{sistema}
				\frac{d \mathcal{L}}{dx}= -\lambda\\
				\frac{d \mathcal{L}}{dy}= -z^2+ \lambda - \mu_1 + \mu_2\\
				\frac{d \mathcal{L}}{dz}= 1- 2 y z\\
				\mu_1 g_1(x,y,z)=\mu_1 y = 0\\
				\mu_2 g_2(x,y,z)=\mu_2 (1-y) = 0 \\
				\mu_1 \ge 0\\
				\mu_2 \ge 0\\
				g_1(x,y,z)=y \ge 0\\
				g_2(x,y,z)=1-y \ge 0\\
				h(x,y,z) = x-y = 0
				
			\end{sistema}
			\]
			
				\item Solve the nonlinear system
				\begin{gather*}
				P(x,y,z,\lambda, \mu_1, \mu_2)= \left\{ x=1, y=1, z=1/2, \lambda = 0, \mu_1 = 0, \mu_2 = 1/4 \right\} 
			\end{gather*}
				\item Verify if the constraints are active
				\begin{gather*}
				g_1(P)=y=1 \ne 0 \Rightarrow \text{It is not an active const. so we drop it}\\
				g_2(P)=1-y=0 \Rightarrow \text{It is an active const.}
			\end{gather*}
				\item Compute  $\nabla g_2(P)$ and $\nabla h(P)$  and put them in a matrix
						\begin{gather*}
				\begin{align*}
					H = \begin{bmatrix}
						\nabla_{\stackrel{-}{x}} g_2(P)\\
						\nabla_{\stackrel{-}{x}} h(P)
					\end{bmatrix} =
					\begin{bmatrix}
						0 & -1 & 0\\
						1 & -1 & 0
					\end{bmatrix}
				\end{align*}
			\end{gather*}
	\end{enumerate}
	}}
	\fbox{ \parbox{\textwidth}{
			\begin{enumerate}\setcounter{enumi}{6}
				\item Find the kernel
				\begin{gather*}
				\begin{align*}
					H v =
					\begin{bmatrix}
						0 & -1 & 0\\
						1 & -1 & 0
					\end{bmatrix} 
					\begin{bmatrix}
						\alpha \\
						\beta \\
						\gamma
					\end{bmatrix} = 0 \\
					k=
					\begin{bmatrix}
						0 \\
						0 \\
						1
					\end{bmatrix} 
				\end{align*}
			\end{gather*}
				\item Write the hessian of the active constraint
				\begin{gather*}
				\nabla ^2 \mathcal{L}(P)=	\begin{bmatrix}
					0 & 0 &  0\\
					0 &  0 & -2z\\
					0 & -2z & -2y \\
				\end{bmatrix}=
				\begin{bmatrix}
					0 & 0 &  0\\
					0 &  0 & -1\\
					0 &-1 & -2 \\
				\end{bmatrix}\\
				\text{Compute }\\
				A=k^T\nabla_{\stackrel{-}{x}}^2\mathcal{L}(P)k=-2 \Rightarrow \text{P is not a minimum}
			\end{gather*}
		\end{enumerate}
	}}
	\fbox{ \parbox{\textwidth}{
			Example of Sylvester's theorem:
			\begin{gather*}
				A=k^T\nabla_{\stackrel{-}{x}}^2\mathcal{L}(P)k =  
				\begin{bmatrix}
					0 & 0 &  0\\
					0 &  2 & 0\\
					0 & 0 & 0 \\
				\end{bmatrix}
			\end{gather*}
			Since $det(A)=0$, then we could not say anything about the nature of the candidate point. We apply the $\varepsilon$ trick\\
			\begin{equation*}
				\begin{bmatrix}
					\centering
					\varepsilon & 0 & 0 \\
					0 & 2+\varepsilon & 0\\
					0 & 0 &\varepsilon
				\end{bmatrix}
			\end{equation*}
			P is a minimum point if all the determinants of the principal minors have $\lim_{\varepsilon \to 0} \frac{d}{d\varepsilon}det(A+\varepsilon I) > 0$
			\begin{gather*}
				det(A\left[ 1 \text{x} 1 \right] )	= det(\varepsilon)=\varepsilon \rightarrow  \lim_{\varepsilon \to 0} \frac{d}{d\varepsilon} \varepsilon = 1 > 0 \Rightarrow OK \\
				det(A\left[ 2 \text{x} 2 \right] )	= 2 \varepsilon + \varepsilon^2 \rightarrow  \lim_{\varepsilon \to 0} \frac{d}{d\varepsilon} 2 \varepsilon + \varepsilon^2 =\lim_{\varepsilon \to 0} 2 + 2 \varepsilon  > 0 \Rightarrow OK\\
				det(A\left[ 3 \text{x} 3 \right] )	= 2 \varepsilon^2 + \varepsilon^3 \rightarrow  \lim_{\varepsilon \to 0} \frac{d}{d\varepsilon} 2 \varepsilon^2 + \varepsilon^3 =\lim_{\varepsilon \to 0} 4 \varepsilon +3 \varepsilon^2 \rightarrow \\
				\rightarrow  \lim_{\varepsilon \to 0} \frac{d}{d\varepsilon} 4 \varepsilon +3 \varepsilon^2 = \lim_{\varepsilon \to 0}4+6\varepsilon > 0 \Rightarrow OK
			\end{gather*}
			So the candidate point is a minimum
	}}
	\newpage
	\section{Functional Minimization}

	minimize F(x,y,z)=$\phi \left( x(a),y(a),z(a),x(b),y(b),z(b) \right)+\int_a^b L(x,y,z,x',y',z',t)dt$
	\begin{gather}
		\text{subject to} \  b_k \left( x(a),y(a),z(a),x(b),y(b),z(b) \right) = 0 \text{ with } k=1,...,m \notag\\
		\ \ \ \ \psi_j\left( x(a),y(a),z(a),x(b),y(b),z(b) \right)+\int_a^b G_j \left( x,y,z,x',y',z',t \right)dt  = 0 \text{ with } j=1,...,p \\
		\text{Introducing} \notag \\
		B\left( x_a,y_a,z_a,x_b,y_b,z_b,\mu_k \right) = \phi \left( x_a,y_a,z_a,x_b,y_b,z_b,\mu_k \right) - \sum_{k=1}^{m} \mu_k b_k\left( x_a,y_a,z_a,x_b,y_b,z_b,\mu_k \right)+ \notag\\
		-\sum_{k=1}^{m} \lambda_j \psi_j\left( x(a),y(a),z(a),x(b),y(b),z(b) \right)\\
		H(x,y,z,x',y',z',\lambda, t)=L(x,y,z,x',y',z',\lambda, t)-\sum_{j=1}^{p}\lambda_j G_j(x,y,z,x',y',z',t)
	\end{gather}
\subsection{Solution with the calculus of the variations}
\fbox{ \parbox{\textwidth}{
		Example of the use of the variations:\\
		\begin{align*}
			\text{minimize} \ \ \ \ \ \  \mathcal {F}(x)=x(0)+\displaystyle \int _0^1 x^2+(x^\prime -t)^2\;\mathrm {d}t\\
			\text{subject to: } \displaystyle \int _0^1 x\;\mathrm {d}t=0\\
			 x(1)=2
		\end{align*}
		Write the lagrangian
		\begin{gather*}
			\mathcal {L}(x,\lambda )=\mathcal {F}(x)-\lambda \displaystyle \int _0^1 x\;\mathrm {d}t - \mu \,(x(1)-2)=\\
			= x(0)+\displaystyle \int _0^1 x^2+(x^\prime -t)^2\;\mathrm {d}t -\lambda \displaystyle \int _0^1 x\;\mathrm {d}t - \mu \,(x(1)-2)
		\end{gather*}
	Perform the variation:
	\begin{gather*}
		\delta (x(0)) + \int_{0}^{1} 2x \delta(x) +2(x'-t) \delta(x')  \ dt - \delta(\lambda)\int_{0}^{1}x \ dt -\lambda \int_{0}^{1} \delta (x) \ dt - \\ 
		+\delta (\mu) \left( x(1) -2 \right) -\mu \delta (x(1))
	\end{gather*}
    Since there is $\delta(x')$ we have to remove it by applying the \textit{integration by part}:
    \begin{gather*}
    	\frac{d \left( \ 2(x'-t) \delta(x) \ \right)}{dt}= 2(x''-1)\delta(x)+2(x'-t)\delta(x') \ ; \\
    	2(x'-t)\delta(x')=\frac{d \left( \ 2(x'-t) \delta(x) \ \right)}{dt}-2(x''-1)\delta(x)
    \end{gather*}
Then we could substitute in the first equation
\begin{gather*}
	\delta (x(0)) + \int_{0}^{1} 2x \delta(x) +\frac{d \left( \ 2(x'-t) \delta(x) \ \right)}{dt}-2(x''-1)\delta(x)  \ dt + \\ 
	- \delta(\lambda)\int_{0}^{1}x \ dt -\lambda \int_{0}^{1} \delta (x) \ dt - \delta (\mu) \left( x(1) -2 \right) -\mu \delta (x(1))
\end{gather*}
Performing the integration
\begin{gather*}
	\delta (x(0)) + \int_{0}^{1} 2x \delta(x)-2(x''-1)\delta(x)  \ dt +\left[ \left( \ 2(x'-t) \delta(x) \ \right) \right] |_0^1 \\ 
	- \delta(\lambda)\int_{0}^{1}x \ dt -\lambda \int_{0}^{1} \delta (x) \ dt - \delta (\mu) \left( x(1) -2 \right) -\mu \delta (x(1))
\end{gather*}
}}

\subsection{Solution with direct writing of the BVP}
	After all the computations, the stationary point (and so the required solution od the exercise) are between the solution of the following system:
	\begin{equation}
	\begin{sistema}
		\text{Euler Lagrange equations:}\\
		\frac{dH}{dx}-\frac{d}{dt}\left(\frac{dH}{dx'}\right)=0\\
		\frac{dH}{dy}-\frac{d}{dt}\left(\frac{dH}{dy'}\right)=0\\
		\frac{dH}{dz}-\frac{d}{dt}\left(\frac{dH}{dz'}\right)=0\\ \\
		\text{Original boundary conditions:}\\
		b_k \left( x(a),y(a),z(a),x(b),y(b),z(b) \right) = 0\\ \\
		\text{Original Integral constraints:}\\
		\psi_j\left( x(a),y(a),z(a),x(b),y(b),z(b) \right)+\int_a^b G_j \left( x,y,z,x',y',z',t \right)dt  = 0\\ \\
		\text{Additional transversal BC}\\
		\frac{dB}{dx_a}-\frac{dH}{dx'}|_a=0 \ \ \frac{dB}{dy_a}-\frac{dH}{dy'}|_a=0 \ \ \ \frac{dB}{dz_a}-\frac{dH}{dz'}|_a=0\\
		\frac{dB}{dx_b}-\frac{dH}{dx'}|_b=0 \ \ \frac{dB}{dy_b}-\frac{dH}{dy'}|_b=0 \ \ \ \frac{dB}{dz_b}-\frac{dH}{dz'}|_b=0\\
	\end{sistema}
	\end{equation}
	
	\newpage \thispagestyle{empty}
	\fbox{ \parbox{\textwidth}{
			\textbf{Example 1:}
			\begin{gather*}
				\text{Minimize: } \phi(x(a),x(b)) + \int_a^b L(x,x',t)dt=\int_0^1 (x')^2 dt\\
				\phi(x(a),x(b))=0\\
				\text{S.T.: } x(0)=-4, \ \ \ x(1)=4, \ \ \ \psi(x(a),x(b))+\int_a^b G(x,x',t)dt=\int_0^1 tx \ dt \\
				\text{with } \psi(x(a),x(b))=0
				\end{gather*}
			\begin{enumerate}
				\item Write the constraints as equations with 0
				\begin{equation*}
				x(0)+4 = 0, \ \ \ x(1)-4=0
				\end{equation*}
				\item Introducing the auxiliary function
				\begin{gather*}
				B(x(a),x(b))=0-\lambda 0 -\mu_1 (x(0)+4)- \mu_2(x(1)-4) \\
				H(x,x',\lambda,t)=L(x,x',t)-\lambda G(x,x',t)=(x')^2-\lambda t x
				\end{gather*}
				\item  Perform the calculation for further use
				\begin{equation*}
				\frac{dH}{dx}=-\lambda t \ \ \ \ \frac{dH}{dx'}= 2 x'
				\end{equation*}
				\item Write the system
			\begin{equation*}
			\begin{sistema}
				\text{Euler Lagrange equations:}\\
				\frac{dH}{dx}-\frac{d}{dt}\left(\frac{dH}{dx'}\right)=-\lambda t-2x''=0\\ \\
				\text{Original boundary conditions:}\\
				x(0)+4 = 0\\ 
				x(1)-4=0 \\ \\
				\text{Original Integral constraints:}\\
				\int_0^1 tx \ dt=0 \\ \\
				\text{Additional transversal BC}\\
				\frac{dB}{dx_a}-\frac{dH}{dx'}|_a=\frac{dB}{dx(0)}-\frac{dH}{dx'}|_{t=0}=-\mu_1 -2x'(0) =0 \\
				\frac{dB}{dx_b}-\frac{dH}{dx'}|_b=\frac{dB}{dx(1)}-\frac{dH}{dx'}|_{t=1}=-\mu_2 -2x'(1) =0 \\ \\
			\end{sistema}
		\end{equation*}
				\item Solve the system
				\begin{gather*}
				\mu_1=-2x'(0) \Rightarrow \text{TRIVIALLY SATISFIED}\\
				\mu_2=-2x'(1) \Rightarrow \text{TRIVIALLY SATISFIED}\\
				x''=-\frac{\lambda}{2}t \rightarrow x=-\frac{\lambda}{12}t^3+c_1t+c_2\\
				x(0)=-4=c_2 \Rightarrow c_2=-4			
				\end{gather*}
\end{enumerate}}}
	\fbox{ \parbox{\textwidth}{
			\begin{gather*}
				x(1)=4=-\frac{\lambda}{12}+c_1+c_2=-\frac{\lambda}{12}+c_-4 \Rightarrow c_1=8+\frac{\lambda}{12}\\
				x(t)=-\frac{\lambda}{12}t^3+\left(8+\frac{\lambda}{12}\right)t-4
				\int_0^1tx \ dt= \\
				 \int_0^1t \left(-\frac{\lambda}{12}t^3+\left(8+\frac{\lambda}{12}\right)t-4\right) \ dt=0 \ ... \ \\
				\frac{\lambda }{4} \frac{1}{15}+1=0 \Rightarrow \lambda=-60
			\end{gather*}
			Finally, the solution is:
			\begin{equation*}
				x(t)=-\frac{\lambda}{12}t^3+\left(8+\frac{\lambda}{12}\right)t-4=5t^3+3t-4
			\end{equation*}
			
			\textbf{Example 2:}
			\begin{gather*}
				\text{Minimize: } \phi(x(a),x(b)) + \int_a^b L(x,x',t)dt=x(1)+\int_0^1 tx+(x')^2 dt\\
				\text{S.T.: } x(0)=1,  \ \ \ 1+\int_0^1 x \ dt=0, \ \ \ x(1)+\int_0^1 tx \ dt =0
			\end{gather*} 
		\begin{enumerate}
				\item Write the constraints as equations with 0
					\begin{gather*}
				x(0)-1 = 0
			\end{gather*}
				\item Introducing the auxiliary function
					\begin{gather*}
				B(x(a),x(b))=x(1)- \lambda_1 \ 1 - \lambda_2 x(1)-\mu (x(0)-1)\\
				H(x,x',\lambda,t)=L(x,x',t)-\lambda_1 G_1(x,x',t)-\lambda_2 G_2(x,x',t)=\\
				=tx+(x')^2-\lambda_1x-\lambda_2tx
			\end{gather*}
				\item Perform the calculation for further use
					\begin{gather*}
				\frac{dH}{dx}=t -\lambda_1 -\lambda_2 t \ \ \ \ \frac{dH}{dx'}= 2 x'
			\end{gather*}
				\item  Write the system
			\begin{gather*} 
			\begin{sistema}
				\text{Euler Lagrange equations:}\\
				\frac{dH}{dx}-\frac{d}{dt}\left(\frac{dH}{dx'}\right)=t -\lambda_1 -\lambda_2 t-2x''=0\\ \\
				\text{Original boundary conditions:}\\
				x(0)-1 = 0\\ 
				\text{Original Integral constraints:}\\
				1+\int_0^1 x \ dt=0\\ 
				x(1)+\int_0^1 tx \ dt =0\\
				\text{Additional transversal BC}\\
				\frac{dB}{dx_a}-\frac{dH}{dx'}|_a=\frac{dB}{dx(0)}-\frac{dH}{dx'}|_{t=0}=-\mu -2x'(0) =0  \\
				\frac{dB}{dx_b}-\frac{dH}{dx'}|_b=\frac{dB}{dx(1)}-\frac{dH}{dx'}|_{t=1}=1- \lambda_2 +2x'(1) =0 \\ \\
			\end{sistema}
		\end{gather*}
	\end{enumerate}
	}}
	\fbox{ \parbox{\textwidth}{
			\begin{enumerate}\setcounter{enumi}{4}
			\item Solve the system
					\begin{gather*}
				\mu=-2x'(0) \Rightarrow \text{TRIVIALLY SATISFIED}\\
				x''=\frac{ \lambda_1 +t( \lambda_2 -1)}{2}=A+Bt \ \ \ \text{with } A=\frac{\lambda_1}{2}, B=\frac{\lambda_2-1}{2}\\
				x=A\frac{t^2}{2}+B\frac{t^3}{2}+c_1 t+ c_2\\
				x(0)=1=c_2 \Rightarrow c_2=1\\
				1-\lambda_2+2x'(1)=0 \Rightarrow c1=\frac{B}{2}-A\\
				x(t)=A t\left(\frac{t}{2}-1\right)+ B \frac{t}{2} \left(\frac{t^3}{3}+1\right)+1\\
				\int_{0}^{1}xt / dt=-1 \rightarrow -\frac{A}{3}+\frac{7}{24}B+1=-1\\
				\int_0^1 tx / dt =-x(1) \rightarrow \frac{A}{2}+\frac{2B}{3}-A+1=-\frac{A}{2}+\frac{2B}{3}+1\\
			\begin{sistema}
				A=\frac{3732}{377}\\
				B=\frac{2983}{377}
			\end{sistema} \Rightarrow 
			\begin{sistema}
				\lambda_1=\frac{7464}{377}\\
				\lambda_2=\frac{2983}{377}
			\end{sistema}
		\end{gather*}
	\end{enumerate}}}
	
	\newpage
	\section{Optimal control}
	\subsection{Boundary time problem}
	\begin{gather*}
		\text{Minimize } F(\stackrel{-}{x},u,p,t)=\phi(x(a),x(b),p)+\int_{a}^{b}L(\stackrel{-}{x},u,p,t)dt  \\\
		\text{subject to: } x'=f(\stackrel{-}{x},u,p,t)\\
		\ \ \ \ \ \ \ \ \ \ \ \ \ \ \ \ \ \ b(x(a),x(b),p)=0\\
		\ \ \ \ \ \ \ \ \ \ \ \ \ \ \ \ \ \ \int_{a}^{b}g_k(\stackrel{-}{x},u,p,t) \ dt = l_k\\
		\ \ \ \ \ \ \ \ \ \ \ \ \ \ \ \ \ \ \ u \in U(t)
	\end{gather*}
	Names:
	\begin{itemize}
		\item p optimization parameter
		\item U(t) control set
		\item $\phi(x(a),x(b),p)$ Mayer term
		\item $L(\stackrel{-}{x},u,p,t)$ Lagrange running cost
		\item $\phi(x(a),x(b),p)+\int_{a}^{b}L(\stackrel{-}{x},u,p,t)dt$ Bolza term - Target of the control
		\item $x'$=f(x,u,p,t) Dynamic system
		\item  b(x(a),x(b),p)=0 Boundary conditions
		\item x = $\left[ x_1 \ x_2 \ ... \ x_n  \right] $ States
		\item u = $\left[ u_1 \ u_2 \ ... \ u_n  \right] $ Controls
	\end{itemize}
	Steps for the solution:\\
	\begin{enumerate}
		\item If the problem is formulated as a maximization problem, then it could be studied as a minimization one by changing the sign
		\begin{gather}
			\text{maximize F($\stackrel{-}{x}$,u,p,t)} \ \ \ \  \Rightarrow \ \ \ \ \text{minimize -F($\stackrel{-}{x}$,u,p,t)}
		\end{gather}
		\item Transform the integral constraint in ODE constraint. FROM NOW ON THE VARIABLE Z BEHAVES AS AN ADJOINED x
		\begin{gather}
			\int_{a}^{b}g_k(\stackrel{-}{x},u,p,t) \ dt = l_k \ \ \ \Rightarrow \ \ \
			\begin{sistema}
				z'_k=g_k(\stackrel{-}{x},u,p,t)\\
				z_k(a)=0\\
				z_k(b)=l_k
			\end{sistema}
		\end{gather}
		\item Transform the optimization parameters in ODE constraints, this transformation does not introduce new boundary conditions
		\begin{gather}
			p' = 0
		\end{gather}
		\item Built the Hemiltonian
		\begin{gather}
			H(\stackrel{-}{x},p,z,\lambda, \mu,\eta,u,t)=L(\stackrel{-}{x},u,p,t)+\lambda f(\stackrel{-}{x},u,p,t)+\mu g(\stackrel{-}{x},u,p,t)+\eta 0
		\end{gather}
		$\eta$ refers to the constraint introduced by p
		\item Built the function B
		\begin{gather}
			B(x(a),x(b),p,\omega)=\phi\left(x(a),x(b),p \right)+\omega b\left(x(a),x(b),p\right)
		\end{gather} 
		\item Write the BVP
		\begin{gather}
			\begin{sistema}
				\text{Original ODE}\\
				\stackrel{-}{x'}=f\left(\stackrel{-}{x},u,p,t \right)\\
				\text{Adjoint equations (remember that also z$\in \stackrel{-}{x}$)}\\
				\lambda_k '=-\frac{\partial H}{\partial x_k }(\stackrel{-}{x},p,\lambda,\mu,u,t)\\
				\text{Integral constraints}\\
				z'_k=g_k(\stackrel{-}{x},u,p,t)\\
				z_k(a)=0\\
				z_k(b)=l_k\\ \\
				\text{Parameters boundary conditions}\\ 
				\frac{\partial B}{\partial p}(x(a),x(b),p,\omega)+\int_{a}^{b}\frac{\partial H}{\partial p }(\stackrel{-}{x},p,\lambda,\mu,u,t)dt=0\\ \\
				\text{Original Boundary conditions}\\
				b(x(a),x(b),p)=0\\ \\
				\text{Adjoint boundary conditions}\\
				\frac{\partial B}{\partial x(a)}(x(a),x(b),p,\omega)+\lambda(a)=0\\
				\frac{\partial B}{\partial x(b)}(x(a),x(b),p,\omega)-\lambda(b)=0\\ \\
				\text{Pontryagin's maximum principle}\\
				u(t)=\text{argmin}\left\{ \ H \left(x(t),p,\lambda(t),\mu,u,t \right) \ \right\} \ \ \ \text{with} \ \ \ u \in U
			\end{sistema}
		\end{gather}
		
		Note: for the Pontryagin's maximum principle if $x_k$ is not bounded in a (or b) , then $\lambda_k$(a)=0 (or $\lambda_k$(b)=0)
	\end{enumerate}
	
	\newpage
	\fbox{ \parbox{\textwidth}{
			\textbf{Example 1:}\\
			\begin{gather*}
				maximize \ \ F(x,y,u,v) = -\left(x(1)y(0) \  + \int_{0}^{1}x(t)^2+u(t)^2 \ dt \right)\\
				s.t. \ \ \ \ x'=u\\
				y'=v+x\\
				\int_{0}^{1}xy+u \ dt =2\\
				x(0)-y(1)=1\\
				u^2+v^2 \le 1
			\end{gather*}
			
			\begin{enumerate}
				\item Write the problem as a minimization one
				\begin{gather*}
					minimize \ \ F(x,y,u,v) = x(1)y(0) + \int_{0}^{1}x(t)^2+u(t)^2 \ dt \\
					s.t. \ \ \ \ x'=u\\
					y'=v+x\\
					\int_{0}^{1}xy+u \ dt =2\\
					x(0)-y(1)=1\\
					u^2+v^2 \le 1
				\end{gather*}
				\item Remove the integral constraints
				\begin{gather*}
					\int_{0}^{1}xy+u \ dt =2 \ \ \ \Rightarrow \ \ \ \begin{sistema}
						z'=xy+u\\
						z(0)=0\\
						z(1)=2
					\end{sistema}
				\end{gather*}
				\item Built the Hemiltonian
				\begin{gather*}
					H(x,y,z,\lambda_1,\lambda_2, \mu,u,v,t)= \left( x^2+u^2 \right)+ \lambda_1 u +\lambda_2 \left( v+x \right)+ \mu \left(xy+u \right)
				\end{gather*}
				\item Built B
				\begin{gather*}
					B(x(0),y(0),z(0),x(1),y(1),z(1),\omega_1,\omega_2,\omega_3)=\\
					x(1)y(0)+\omega_1 \left( \ x(0)-y(1)-1 \ \right)+\omega_2 \left( \ z(0)-0 \ \right)+ \omega_3 \left( \ z(1)-2 \ \right)
				\end{gather*}
	\end{enumerate}}}
	
	\fbox{ \parbox{\textwidth}{
			\begin{enumerate}\setcounter{enumi}{4}
				\item Built the BVP
				\begin{gather*}
					\begin{sistema}
						\text{Original ODE}\\
						x'=u\\
						y'=v+x\\
						\text{Adjoint equations (remember that also z$\in \stackrel{-}{x}$)}\\
						\lambda_1 '=-\frac{\partial H}{\partial x }=-2x-\lambda_2-y\mu \\
						\lambda_2'=-\frac{\partial H}{\partial y }=-x \mu \\
						\mu'=-\frac{\partial H}{\partial z }=0 \\
						\text{Integral constraints}\\
						z'=xy+u\\
						z(0)=0\\
						z(1)=2\\ \\
						\text{Original Boundary conditions}\\
						x(0)-y(1)=1\\ \\
						\text{Adjoint boundary conditions}\\
						\frac{\partial B}{\partial x(0)}+\lambda_1(0)=0 \ \ \ \ \rightarrow \ \ \ \ \  \omega_1 +\lambda_1(0)=0\\
						\frac{\partial B}{\partial y(0)}+\lambda_2(0)=0 \ \ \ \ \rightarrow \ \ \ \ \ x(1) +\lambda_2(0)=0 \\
						\frac{\partial B}{\partial z(0)}+\mu(0)=0 \ \ \ \ \rightarrow \ \ \ \ \ \omega_2 +\mu(0)=0 \\ 
						\frac{\partial B}{\partial x(1)}-\lambda_1(1)=0 \ \ \ \ \rightarrow \ \ \ \ \  y(0)-\lambda_1(1)=0\\
						\frac{\partial B}{\partial y(1)}-\lambda_2(1)=0 \ \ \ \ \rightarrow \ \ \ \ \ -\omega_1 -\lambda_2(1)=0 \\
						\frac{\partial B}{\partial z(1)}-\mu(1)=0 \ \ \ \ \rightarrow \ \ \ \ \ \omega_3 -\mu(1)=0 \\ \\
						\text{Pontryagin's maximum principle}\\
						\left( u(t), v(t) \right)=\text{argmin}\left\{ \ H \left(x(t),y(t),z(t),\lambda_1(t),\lambda_2(t),\mu,u,v,t \right) \ \right\}= \\
						= \ \text{argmin}\left\{ \ x^2+u^2+\lambda_1 u + \lambda_2 (v+x) +\mu (xy+u) \ \right\}\\
						\text{with} \ \ \ \ u^2+v^2 \le 1 					
					\end{sistema}
				\end{gather*}
				Since not all the components of H depend on u or v then we can consider only:
				\begin{gather*}
					\text{minimize} \ \ \ \stackrel{\sim}{H} = u^2+\lambda_1 u + \lambda_2 v +\mu u\\
					\text{s.t.}	\ \ \ \ u^2+v^2 \le 1
				\end{gather*}
				\item From now the problem could be solved as a constrained minimization (for example with the KKT method)
			\end{enumerate}
	}}
	\newpage
	\subsection{Free time problem}
	\begin{gather*}
		\text{Minimize } F(x,u,t)=\phi(x(0),x(T))+\int_{0}^{T}L(x,u,t)dt\\
		\text{subject to: } x'=f(x,u,t)\\
		\ \ \ \ \ \ \ \ \ \ \ \ \ \ \ \ \ \ b(x(0),x(T))=0\\
		\ \ \ \ \ \ \ \ \ \ \ \ \ \ \ \ \ \ \ u \in U(t)
	\end{gather*}
	\begin{enumerate}
		\item If the problem is formulated as a maximization problem, then it could be studied as a minimization one by changing the sign
		\begin{gather*}
			\text{maximize F(x,u,t)} \ \ \ \  \Rightarrow \ \ \ \ \text{minimize -F(x,u,t)}
		\end{gather*}
		\item Perform a change of variable
		\begin{gather*}
			t=sT\\
			\stackrel{\sim}{x}(s)=x(sT)\\
			\stackrel{\sim}{u}(s)=u(sT)\\
			\stackrel{\sim}{x'}(s)=Tf(\stackrel{\sim}{x},\stackrel{\sim}{u},Ts)
		\end{gather*}
		\item The problem is now transformed in
		\begin{gather*}
			\text{minimize } \phi(\stackrel{\sim}{x}(0),\stackrel{\sim}{x}(1)) +\int_{0}^{1} T L(\stackrel{\sim}{x},\stackrel{\sim}{u},Ts) \ ds \\
			\text{s.t.} \stackrel{\sim}{x'}(s)=Tf(\stackrel{\sim}{x},\stackrel{\sim}{u},Ts) \\
			\ \ \ \ \ \ b(\stackrel{\sim}{x}(0),\stackrel{\sim}{x}(1))=0\\
			T'(s)=0
		\end{gather*}
		From now on we could use the same steps for the solution of the boundary time problem. For the simplicity of the notation we will use x and u instead of  $\stackrel{\sim}{x}$ and $\stackrel{\sim}{u}$. \\
		Note: T is now a variable and not a parameter so there is also an adjoint equation of:
		\begin{equation*}
			\lambda = -\frac{\partial H}{\partial T}
		\end{equation*}
		If a variable has a boundary constraint with T, then it must be changed as in the following example
		\begin{gather*}
			\int_{0}^{T}...dt \ \ \  s.t. \ \ \ \  y(T)=1 \ \ \ \ \Rightarrow \ \ \ \ \ \ 	\int_{0}^{1}...ds \ \ \  s.t. \ \ \ \ \ y(1)=1 
		\end{gather*}
	\end{enumerate}
	\newpage
	\fbox{ \parbox{\textwidth}{
			\textbf{Example 2: Change of variable for the free time problem}
			\begin{gather*}
				minimize \ \ F = \int_{0}^{T}(1+t^2)u^2+y \ dt \\
				s.t. \ \ \ \ x'=y+u\\
				y'=xy\\
				x(0)=0\\
				y(T)=1
			\end{gather*}
			The problem could be transformed in
			\begin{gather*}
				minimize \ \ F = \int_{0}^{1}T\left[ (1+(Ts)^2)u^2+y \ \right] ds  \\
				s.t. \ \ \ \ x'=T(y+u)\\
				y'=T(xy)\\
				x(0)=0\\
				y(1)=1\\
				T'(s)=0
			\end{gather*}
	}}
	
\end{document}
